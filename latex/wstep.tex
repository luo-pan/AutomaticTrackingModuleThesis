\chapter{Wprowadzenie}
\label{cha:wprowadzenie}

%---------------------------------------------------------------------------

\section{Cele pracy}
\label{sec:celePracy}

%---------------------------------------------------------------------------

\section{Zakres pracy}
\label{sec:zakresPracy}

\section{Śledzenie obiektów}
\label{sec:Sledzenie_obiektow}
Systemy realizujące zadanie śledzenia obiektów, rozumiane jako ciągłą obserwację zmiany stanu obiektu określonego wcześniej jako obiekt zainteresowania, charakteryzują się dużą złożonością i interdyscyplinarnością. W odniesieniu do robotyki mobilnej, w literaturze pod pojęciem ,,śledzenie obiektów'' (ang. \textit{object tracking}) rozumie się:

\begin{enumerate}

	\item \label{itm:Algorytm_przetwarzania_obrazow} Funkcję algorytmu cyfrowego przetwarzania obrazów, polegającą na wyróżnieniu obiektu zainteresowania, określeniu jego charakterystycznych właściwości oraz uaktualnieniu informacji o jego położeniu na kolejnych klatkach sekwencji video

	\item \label{itm:System_sterowania} Funkcję systemu sterowania robota, dokonującego rejestracji obecności obiektu zainteresowania oraz zmiany stanu robota pozwalającej na utrzymanie tego obiektu w zasięgu pokładowych sensorów, ewentualnie przy jednoczesnej realizacji określonego zadania, np. jak największym zbliżeniu się do obiektu
	
\end{enumerate}

Poniżej dokonano obszerniejszej analizy tych dwóch grup zagadnień.

\todo{Opisać powiązanie poprzez Visual Servoing}








