\chapter{Wprowadzenie}
\label{cha:Wprowadzenie}

Wizyjne śledzenie obiektów od lat stanowi przedmiot intensywnych badań, motywowanych jego szerokim zastosowaniem praktycznym, m. in. w systemach automatycznego monitoringu, interfejsach człowiek-maszyna oraz zaawansowanej nawigacji. W niniejszej pracy przedstawiono projekt modułu wizyjnego śledzenia obiektów dla robotyki mobilnej. Zaprojektowanie wspomnianego modułu wymagało zapoznania się i przedstawienia obszernej literatury przedmiotu, która reprezentuje różne dziedziny wiedzy, przedstawione w części poświęconej zakresowi pracy. Przegląd dokonanych w tej sprawie ustaleń został zawarty w rozdziałach od drugiego do czwartego, z których pierwszy przybliża ogólną teorię śledzenia obiektów, drugi omawia powiązane zagadnienia z dziedziny cyfrowego przetwarzania obrazów, a trzeci dostarcza szczegółowego opisu algorytmów spotykanych w robotyce mobilnej. W rozdziale czwartym przedstawiono przegląd istniejących rozwiązań systemów wizyjnego śledzenia obiektów w robotyce mobilnej. Na podstawie dokonanego przeglądu literatury anglojęzycznej, w związku z licznymi i różnorodnymi rozwiązaniami, jakie pojawiają się w tej dziedzinie trzeba było dokonać wyboru tych, które zostaną rozwinięte w niniejszej pracy. W efekcie tego wyboru praca ta omawia dwa równorzędne warianty, zwane modułami, pierwszy oparty na algorytmie Lucasa-Kanade'a i drugi oparty na algorytmie \textit{CAMShift}. Zostały one przedstawione odpowiednio w rozdziale siódmym i ósmym. Wyczerpujący opis powodów, dla których wybrano te, a nie inne algorytmy znajduje się w rozdziale szóstym. Rozdział ten zawiera ogólne wymagania projektowe dotyczące implementowanych modułów, takich samych w obydwu wypadkach, choć różnie w nich zrealizowanych. Na podstawie tych wymagań skonfrontowanych ze stanem wiedzy, opisanym w literaturze, wybrano wspomniane wyżej algorytmy, które spełniają założenia projektowe, o jednocześnie zaawansowanym i podstawowym charakterze.

\section{Cele pracy}
\label{sec:Cele_pracy}

Celem pracy jest zaprojektowanie modułu wizyjnego śledzenia obiektów, który spełnia wymagania wynikające z zastosowań w robotyce mobilnej. Dla jego realizacji zostały opracowane dwa wariantowe rozwiązania, oparte na algorytmie Lucasa-Kanade'a oraz algorytmie \textit{CAMShift}. Moduł wizyjny służy do estymacji kolejnych położeń wybranego obiektu zainteresowania w sekwencji video i może stanowić użyteczne źródło danych wejściowych dla zaawansowanych systemów sterowania robotów mobilnych. 

\section{Zakres pracy}
\label{sec:Zakres_pracy}

Praca obejmuje następujące obszary badawcze:

\begin{enumerate}

	\item Teorię śledzenia obiektów
	\item Cyfrowe przetwarzanie obrazów, w tym:
	
	\begin{enumerate}
	
		\item Ekstrakcję i deskrypcję cech obiektów
		\item Analizę ruchu w sekwencjach video, w szczególności estymację przepływu optycznego
		
	\end{enumerate}		
	
	\item Teorię sterowania, w szczególności teorię estymatorów stanu
	\item Statystykę nieparametryczną, w szczególności metody jądrowej estymacji gęstości prawdopodobieństwa
	

\end{enumerate}